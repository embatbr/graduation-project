\chapter{Speaker Recognition Systems}

The process of voice recognition lies on the field of pattern classification, with the speaker and his or her utterance (a speech signal) as inputs for a classifier and a decision as output. This decision may be, given a utterance $\boldsymbol{Y}$ produced by a speaker $\mathcal{S}$ and a set $\boldsymbol{\mathcal{S}} = \{\mathcal{S}_1, ..., \mathcal{S}_S\}$ of known users,

\begin{equation}
    \mathcal{S} = \arg\max_i P(\mathcal{S}_i|\boldsymbol{Y}).
    \label{eq:decision_speaker_identification}
\end{equation}

\noindent This is a case of speaker identification and the output is a $\mathcal{S}$ from $\boldsymbol{\mathcal{S}}$. Another type of decision is

\begin{equation}
    \mathcal{S} == \mathcal{S}_i, \text{ if } P(\mathcal{S}_i|\boldsymbol{Y}) \geq \alpha.
    \label{eq:decision_speaker_verification}
\end{equation}

\noindent This is a speaker verification decision, with a binary output (true or false, accept or reject, belongs or not to the class $\mathcal{S}_i$), given a $\mathcal{S}$ who produced $\boldsymbol{Y}$, a claimed identity $\mathcal{S}_i$ from $\boldsymbol{\mathcal{S}}$ and a threshold $\alpha$ for acceptance. This chapter (and indirectly the whole document) is about the type of decision from \equationref{decision_speaker_verification}.

\section{Basic Concepts}

Before continuing, it is necessary to clarify important basic concepts for speaker verification, such as \textbf{utterance}, \textbf{features} and \textbf{dependency} of the spoken text.

\subsection{Utterance}

An utterance is a piece of speech produced by a speaker. It may be a word, a statement or any vocal sound. The terms \emph{utterance} and \emph{speech signal} sometimes are used interchangeably, but here speech signal will be associated to an utterance recorded and digitalized. An example of utterances as speech signals is shown in \figureref{speech_signals}.

\begin{figure}[ht]
    \centering
    \includegraphics[width=\textwidth]{speech_signals}
    \caption{Speech signals for utterances ``karen livescu" (SE), ``mint chocolate chip" (SD), ``carol owens" (IE) and ``mary miller" (ID).}
    \label{fig:speech_signals}
\end{figure}

\subsection{Features}

\subsection{Dependency x Independency}

\section{Likelihood Ratio Test}

TODO basear-se na seção 2 do artigo ``Speaker Verification Using Adapted Gaussian Mixture Models".

\section{Basic Speaker Verification Architecture}

\subsection{Training Phase}

\subsection{Test Phase}