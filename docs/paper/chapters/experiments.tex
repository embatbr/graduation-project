\chapter{Experiments}
\label{ch:experiments}

This chapter details the experiments performed on the systems described in the previous chapters, contemplating from the front-end processes until the speaker modeling and the log-likelihood ratio test (see \equationref{score_of_X}). First, a description of the corpus is made. Then, explanations about the implementation are given. At last, the results are exhibited using the feature extraction process and the GMM techniques.

\section{Corpus}
\label{sec:corpus}

The database used in this work is \emph{The MIT Mobile Device Speaker Verification Corpus} (MIT-MDSCV), \refbib{Woo et. al.}{woo.park.hazen.2006}, a \textbf{corpus} designed to evaluate voice biometric systems of high mobility. All utterances were recorded using mobile devices of different models and manufacturers.

This corpus is composed of three sections. The first, named ``Enroll 1", contains 48 speakers (22 females and 26 males), each with 54 utterances (names of ice cream flavors) of 1.8 seconds average duration, and is used to train the ASR system. The utterances were recorded in three different locations (a quiet office, a mildly noisy hallway, and a busy street intersection) as well as two different microphones (the built-in internal microphone of the handheld device and an external earpiece headset) leading to 6 distinct training conditions. The second section, named ``Enroll 2", is similar to the first with a difference in the order of the spoken utterances, and is used to test the enrolled speakers. The third section, named ``Imposters" is similar to the first two, but with 40 non-enrolled speakers (17 females and 23 males), and is used to test the robustness of the ASR system. \tableref{corpus-division} summarizes the division of the corpus.

\begin{table}[h]
    \centering
    \begin{tabular}{|l|c|c|}
    \hline
    \multicolumn{1}{|c|}{{\bf Section}} & {\bf Training} & {\bf Test} \\ \hline
    Enroll 1                            & {\bf X}        & {\bf }     \\ \hline
    Enroll 2                            & {\bf }         & {\bf X}    \\ \hline
    Imposter                            & {\bf }         & {\bf X}    \\ \hline
    \end{tabular}
    \caption{Corpus divided in training and test sets.}
    \label{tab:corpus-division}
\end{table}

All utterances are recorded in uncompressed WAV files using a single channel. For each utterance record there is a correspondent text file containing pertinent information, such as speaker who produced it, microphone used, location where the utterance was recorded, the spoken message content and etc (see \tableref{utterance-info}).

Despite being a base for speaker verification systems, in this paper MIT-MDSCV is also used for speaker identification experiments. The difference is that only ``Enroll 1" and ``Enroll 2" are used, for training and test respectively. In an ideal identification system all utterances from ``Enroll 2" are correctly identified and, in an ideal verification system there are no false detection and false rejection.

\begin{table}[h]
    \centering
    \begin{tabular}{|l|c|}
    \hline
    {\bf Speaker}    & f00              \\ \hline
    {\bf Session}    & 1                \\ \hline
    {\bf List}       & female\_list\_3a \\ \hline
    {\bf Gender}     & female           \\ \hline
    {\bf Location}   & Office           \\ \hline
    {\bf Microphone} & Headset          \\ \hline
    {\bf Phrase}     & karen livescu    \\ \hline
    \end{tabular}
    \caption{Information from first utterance of first speaker of session ``Enroll 1".}
    \label{tab:utterance-info}
\end{table}

\section{Coding and Data Preparation}
\label{sec:coding-and-data-preparation}