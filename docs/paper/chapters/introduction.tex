\chapter{Introduction}

The rise in popularity and naturality of computational systems in the everyday of modern life creates the need for easy and non invasive forms of authentication. While enter a hard to memorize password in a terminal still is the safest approach, voice biometrics presents itself as an alternative with continuing improvement. Also, speech is the most natural way humans have to communicate, being incredibly complex and with numerous specific details related to the person who produces it \autocite{bimbot.et.al.2004}. Therefore, it is expected an increasing usage of vocal interfaces to perform actions such as authenticate in a system, command a machine, and identify who is talking and the content of the conversation.

Over the last decade, voice recognition technology has appeared in many commercial products (e.g. Google Now and Apple Siri) with relatively high popularity. Most of those are intended to perform speech recognition (recognize the content of the speech) instead of speaker recognition (recognize who is producing the input speech). To achieve this goal, techniques in Natural Language Processing (NLP) have become popular. This topic is beyond the scope of this paper, and what will be covered in the rest of the document is a subset of speaker recognition.

\section{Speaker Recognition}

TODO explicar com mais detalhes o que é reconhecimento de speaker e descrever seus subtipos (identification e verification, tanto dependentes quanto independentes de texto (tema deste trabalho)).

\section{Objectives}

The objectives of this study are:

\begin{itemize}\itemsep0pt
    \item Analyse and evaluate the speaker verification system using adapted GMM proposed by Reynolds et al. \autocite{reynolds.quatieri.dunn.2000};
    \item Propose and evaluate a new method derived from GMM, using the FCM theory proposed by Gao et al. \autocite{gao.zhou.pu.2013};
    \item Conduct experiments and validation of the existent and the proposed methods.
\end{itemize}

\section{Document Structure}

Chapter 2 contains a brief historical context and some basic details about voice recognition, as well as the basic architecture for a speaker verification system. The feature extraction process is explained in chapter 3, from the reasons for its use to the chosen technique (MFCC). In chapter 4 is detailed the GMM and the UBM-GMM. Chapter 5 introduces FCM and the proposed FGMM. Experiments are described in chapter 6, as well as its results. Finally, chapter 7 concludes the study. Furthermore, this work contains an appendix with the most relevant pieces of the source code and some mathematics concepts used.