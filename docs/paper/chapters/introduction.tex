\chapter{Introduction}
\pagenumbering{arabic}

The rise in popularity and naturality of computational systems in the everyday of modern life creates the need for easy and less invasive forms of authentication. While enter a hard to memorize password in a terminal still is the safest approach, voice biometrics presents itself as a continuing improvement alternative. Also, speech is the most natural way humans have to communicate, being incredibly complex and with numerous specific details related to the speaker \autocite{bimbot.et.al.2004}. Therefore, it is expected an increasing usage of vocal interfaces to perform actions such as computer login, voice search (e.g. Siri, Google Now and Samsung S Voice) and identification of speakers in a conversation and its content.

At present, fingerprint biometrics is adopted in several solutions, e.g. Automated Teller Machine (ATMs) \autocite{wang.wu.2002}, authentication through facial recognition comes as built-in software for average computers and iris scan was adopted for a short time by United Kingdom and permanently by United Arab Emirates border controls \autocite{sasse.2007, raisi.khouri.2008}. Improvements in voice recognition techniques indicate a near future where vocal commands will be used for authentication, entirely or combined with other methods.

Current commercial products based on voice technology are usually intended to perform either \textbf{speech recognition} (\emph{what} is being said) or \textbf{speaker recognition} (\emph{who} is speaking). Voice search applications are designed to determine the content of a speech, with no concern about who the speaker is or if there is more than one, while computer log-in and telephone fraud prevention supplement a memorized personal identification code with speaker verification \autocite{reynolds.1995}. Few applications need to perform both processes, such as automatic speaker labeling of recorded meetings, that transcribes what each person is saying. To achieve this goal, numerous voice processing techniques have become popular, e.g. Natural Language Processing (NLP), Hidden Markov Models (HMM) and GMMs. Although all of these are state-of-the-art, the subject covered by this paper is a subarea of speaker recognition and only a small subset of techiniques will be unraveled.

\section{Speaker Recognition}

As stated in \autocite{pinheiro.2013}, speaker recognition may be divided in two subareas. The first is \textbf{speaker identification}, aimed to determine the identity of a speaker (through a speech signal) from a non-unitary set of known speakers. This task is also named speaker identification in \textbf{closed set}. In the second, \textbf{speaker verification}, the goal is to determine if a speaker is who he or she claims to be, not an imposter. As the set of imposters is unknown \emph{a priori}, this is an \textbf{open set} problem. An intermediate task is \textbf{speaker identification in open set}, when an ``imposter class" is added to the system in order to categorize all unmatched speakers found.

Restrictions on the type of text may be used. In \textbf{text-dependent} systems the content of the speech is relevant to the evaluation, and the training and test utterances must contain the same text (but not necessarily the same intonation), e.g. a passphrase. \textbf{Text-independent} systems have no restrictions to the message in both sets, with the non-textual characteristics of the user's voice (e.g. pitch) being the important aspects to the evaluator. These characteristics are presented in different sentences, usage of different languages and even in gibberish.

This paper is focused in \textbf{text-independent speaker verification}, in other words, the determination of a user's claimed identity by analysis of his or her vocal characteristics with no specific text. To achieve that, a speaker's GMM adapted from an UBM \autocite{reynolds.quatieri.dunn.2000} is implemented. Also, an adaptation of the technique is proposed and evaluated using the theory of FCM presented in \autocite{gao.zhou.pu.2013}.

\section{Objectives}

The objectives of this study are:

\begin{itemize}\itemsep0pt
    \item Analyze and evaluate the speaker verification system using the adapted GMM presented in \autocite{reynolds.quatieri.dunn.2000};
    \item Propose and evaluate a new method derived from GMM, using the FCM theory presented in \autocite{gao.zhou.pu.2013};
    \item Conduct experiments for the existent and the proposed methods and run comparisons.
\end{itemize}

\section{Document Structure}

Chapter 2 contains basic information about voice recognition, as well as the basic architecture for a speaker verification system. The feature extraction process is explained in chapter 3, from the reasons for its use to the chosen technique (MFCC). In chapter 4 is detailed the GMM and the UBM-GMM. Chapter 5 introduces FCM and the proposed FGMM. Experiments are described in chapter 6, as well as its results. Finally, chapter 7 concludes the study. Furthermore, this work contains an appendix with the most relevant pieces of the source code and some necessary mathematical concepts.