\documentclass{gradpaper}


\university{Universidade Federal de Pernambuco}
\institute{Centro de Informática}
\address{Recife}
\degree{Graduate}
\majorfield{Computer Engineering}
\title{Improvements in Speaker Recognition using Fractional Covariance Matrix}
%\date{TODO colocar a data da defesa}
\author{Eduardo Martins Barros de Albuquerque Tenório}
\adviser{Tsang Ing Ren}
\advisertitle{PhD}
\reviser{George Darmiton da Cunha Cavalcanti}
\revisertitle{PhD}


\begin{document}


% Pre-text

\frontpage
\presentationpage
\signatures

\acknowledgements{I am thankful to my parents, for the support and patience
during this long graduation,
\\To my adviser, Tsang Ing Ren, for the guidance,
\\To the colleagues Hector Pinheiro and Sérgio Renan Vieira, for helping me to understand the
GMM.
}

\begin{epigraph}[nota]{autor}
TODO
\end{epigraph}

\abstract
TODO abstract
\begin{keywords}
TODO keywords
\end{keywords}

\listoffigures
\listoftables
%\listofcodes
\tableofcontents


% Text

% TODO colocar aqui a chamada para os capítulos (1-intro, 2-feature-extraction,
% 3-GMM, 4-FCM, 5-experimentos, 6-conclusão)


% Post-text

%\appendix
%
%
%\nocite{*}
%\bibliographystyle{alpha}
%\bibliography{biblio}
%
%
%\colophon

\end{document}