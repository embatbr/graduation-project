\documentclass[12pt,twoside,openright]{report}

%\usepackage{mathptmx}
\usepackage[utf8]{inputenc}
\usepackage[english]{babel}
\usepackage[pdftex]{graphicx}
\usepackage[a4paper,width=150mm,top=25mm,bottom=25mm,bindingoffset=6mm]{geometry}
\usepackage{fancyhdr}
\usepackage[backend=bibtex,sorting=none,citestyle=numeric-comp]{biblatex}
\usepackage{mathtools}
\usepackage{amsmath}
\usepackage[font={small}]{caption}
\usepackage{bm}
\usepackage{titlesec}
\usepackage{emptypage}
\usepackage{times}
\usepackage{csquotes}
\usepackage{multirow}
\usepackage{array}
\usepackage{etoolbox}
\usepackage{algorithm2e}
\usepackage{algpseudocode}
\usepackage{pifont}


% PRE CONFIGURATIONS

\titleformat{\chapter}
  {\normalfont\LARGE\bfseries}{\thechapter.}{1em}{}
\graphicspath{{images/}}
\addbibresource{references.bib}
\setlength{\headheight}{14.5pt}
\pagestyle{fancy}
\fancyhead[RO,LE]{}
\fancyhead[LO]{\thechapter.\space\leftmark}
\fancyhead[RE]{\rightmark}
\renewcommand{\chaptermark}[1]{\markboth{\uppercase{#1}}{}}
\bibliography{references}
\DefineBibliographyStrings{english}{%
  bibliography = {References},
}

\AtBeginEnvironment{tabular}{\footnotesize}

\renewcommand{\figurename}{\small Figure}
\newcommand{\figureref}[1]{Fig. \ref{fig:#1}}
\newcommand{\figurerefcomp}[1]{\ref{fig:#1}}
\newcommand{\equationref}[1]{Eq. \ref{eq:#1}}
\newcommand{\equationrefcomp}[1]{\ref{eq:#1}}
\newcommand{\chapterref}[1]{Chap. \ref{ch:#1}}
\newcommand{\chapterrefcomp}[1]{\ref{ch:#1}}
\newcommand{\sectionref}[1]{Sec. \ref{sec:#1}}
\newcommand{\sectionrefcomp}[1]{\ref{sec:#1}}
\newcommand{\appendixref}[1]{Apx. \ref{apx:#1}}
\newcommand{\appendixrefcomp}[1]{\ref{apx:#1}}
\newcommand{\tableref}[1]{Tab. \ref{tab:#1}}
\newcommand{\tablerefcomp}[1]{\ref{tab:#1}}
\newcommand{\refbib}[2]{\textit{#1} \cite{#2}}

\newcolumntype{M}[1]{>{\centering\arraybackslash}m{#1}}

%MATH COMMANDS
\newcommand{\dvec}[1]{\boldsymbol{#1}}
\newcommand{\dhatvec}[1]{\boldsymbol{\hat{#1}}}
\newcommand{\prob}[1]{P(#1)}
\newcommand{\postprob}[2]{P(#1|#2)}
\newcommand{\pdf}[1]{p(#1)}
\newcommand{\pdfi}[1]{p_i(#1)}
\newcommand{\postpdf}[2]{p(#1|#2)}
\newcommand{\postpdfi}[2]{p_i(#1|#2)}
\newcommand{\determinant}[1]{|#1|}
\newcommand{\dgaussian}[3]{\frac{1}{(2\pi)^{D/2}\determinant{\dvec{#3}}^{1/2}}e^{-\frac{1}{2}(\dvec{#1} - \dvec{#2})'\dvec{#3}^{-1}(\dvec{#1} - \dvec{#2})}}
\newcommand{\dgaussianmixture}{\sum_{i=1}^M w_i \frac{1}{(2\pi)^{D/2}\determinant{\dvec{\Sigma}_i}^{1/2}}e^{-\frac{1}{2}(\dvec{x} - \dvec{\mu}_i)'\dvec{\Sigma}_i^{-1}(\dvec{x} - \dvec{\mu}_i)}}
\newcommand{\verifytest}[2]
{
    \left\{
        \begin{array}{ll}
            \geq #1, & \text{accept } #2,\\
            < #1, & \text{reject } #2.
        \end{array}
    \right.
}
\newcommand{\verifytestB}[2]
{
    \left\{
        \begin{array}{ll}
            \geq #1, & \text{accept } #2,\\
            < #1, & \text{reject } #2,
        \end{array}
    \right.
}


%GLOBAL DEFINITIONS

\gdef\universityname{Universidade Federal de Pernambuco}
\gdef\centername{Centro de Informática}
\gdef\programname{Graduação em Engenharia da Computação}
\gdef\papertitle{Text-Independent Speaker Recognition Using Gaussian Mixture Models}
\gdef\papertype{Dissertation in Computer Engineering}
\gdef\authorname{Eduardo Martins Barros de Albuquerque Tenório}
\gdef\advisername{Tsang Ing Ren}
\gdef\adviserfullname{Prof. Dr. \advisername}
\gdef\revisername{George Darmiton da Cunha Cavalcanti}
\gdef\reviserfullname{Prof. Dr. \revisername}
\gdef\defensedate{Recife, \today}%June 20, 2015}


\title{
    {\includegraphics[width=0.5\textwidth]{cinlogo.png}}
    \\
    {\normalsize\textbf\universityname}
    \\
    {\normalsize\textbf\centername}
    \\
    {\normalsize\textbf\programname}
    \vfill
    \textbf{\papertitle}
    \vskip\baselineskip
    {\large \papertype}
    \vfill
}
\author{\normalsize\authorname}
\date{\normalsize\vfill \defensedate}


\begin{document}
\pagenumbering{gobble}
\maketitle


%FRONTMATTER

\chapter*{Declaration}
This paper is a presentation of my research work, as partial fulfillment of the requirement for the degree in Computer Engineering. Wherever contributions of others are involved, every effort is made to indicate this clearly, with due reference to the literature, and acknowledgement of collaborative research and discussions.
\vskip\baselineskip
\noindent The work was done under the guidance of \adviserfullname\space and was revised by \reviserfullname, at Centro de Informática, Universidade Federal de Pernambuco, Brazil.

\vskip3\baselineskip
\begin{flushright}
\rule{0.75\textwidth}{1pt}
\vskip0.5\baselineskip
\authorname
\end{flushright}

\vskip2\baselineskip
\noindent In my capacity as supervisor of the candidate’s paper, I certify that the above statements are true to the best of my knowledge.

\vskip3\baselineskip
\begin{flushright}
\rule{0.75\textwidth}{1pt}
\vskip0.5\baselineskip
\adviserfullname
\end{flushright}

\vskip2\baselineskip
\noindent In my capacity as revisor of the candidate’s paper, I certify that the above statements are true to the best of my knowledge.

\vskip3\baselineskip
\begin{flushright}
\rule{0.75\textwidth}{1pt}
\vskip0.5\baselineskip
\reviserfullname
\end{flushright}

\vfill
\begin{center}
\defensedate
\end{center}

\chapter*{Acknowledgements}
I am thankful to my family, for the support and patience during the graduation,\\
To my adviser, Tsang Ing Ren, for the guidance,\\
To Cleice Souza, for the previous readings and suggestions,\\
To Sérgio Vieira, Hector Pinheiro and James Lyons, for clarify many of my questions.

\chapter*{}
\vfill
\begin{flushright}
    \textit{Live long and prosper}\\
    \vskip0.5\baselineskip
    Vulcan salute
\end{flushright}
\vfill

\chapter*{Abstract}
TODO escrever o abstract após terminar tudo (após a conclusão).\\

\tableofcontents
%\pagenumbering{roman}


%MIDDLEMATTER

\section{Introdução}
\label{sec:intro}

\contentscurrent

\subsection{Reconhecimento de Locutor}

\begin{frame}
\frametitle{Reconhecimento de Locutor}
\begin{description}
    \item[Identificação] Determina a identidade de um locutor dentro de um conjunto não unitário
    \pause
    \begin{itemize}
        \item 1 para N
        \item Problema de \textbf{conjunto fechado}
        \pause
    \end{itemize}
    \item[Verificação] Determina se o locutor é quem diz ser
    \pause
    \begin{itemize}
        \item 1 para 1
        \item Problema de \textbf{conjunto aberto}
        \pause
    \end{itemize}
\end{description}

\begin{figure}
    \centering
    \includegraphics[width=0.75\textwidth]{speaker-recognition}
\end{figure}
\end{frame}

\begin{frame}
\frametitle{Dependência de texto}
\begin{description}
    \item[Com] Teste $\in$ Treinamento
    \pause
    \begin{itemize}
        \item Diversos graus de dependência
        \item Teste $\not\in$ Treinamento $\implies$ Retreinamento
        \pause
    \end{itemize}
    \item[Sem] Teste $\neq$ Treinamento
    \pause
    \begin{itemize}
        \item Características não textuais
        \item Presentes em diferentes sotaques e até \emph{gibberish}
        \pause
    \end{itemize}
    \item Este trabalho é focado em \textbf{reconhecimento de locutor independente de texto}
\end{description}
\end{frame}

\subsection{Modelos de Mistura Gaussiana}

\begin{frame}
\frametitle{Modelos de Mistura Gaussiana}
\begin{description}
    \item[GMM] \textbf{Combinação} de Gaussianas
    \item[UBM] GMM gerado por diversas \textbf{locuções de fundo}
    \item[AGMM] GMM \textbf{adaptado} a partir de um UBM
    \item[FGMM] GMM \textbf{fracionário} utilizando FCM
\end{description}
\end{frame}

\subsection{Objetivos}

\begin{frame}
\frametitle{Objetivos}
\begin{description}
    \item Implementar sistemas de reconhecimento de locutor e analizar:
    \pause
    \begin{itemize}
        \item Taxas de \textbf{sucesso} para identificação
        \pause
        \begin{itemize}
            \item Diferentes tamanhos de mistura ($M$)
            \item Diferentes tamanhos de características ($\boldsymbol{\Delta}$)
            \pause
        \end{itemize}
        \item Comparar identificações utilizando GMM e FGMM
        \pause
        \item Taxas de \textbf{falsa detecção} e \textbf{falsa rejeição} para verificação
        \pause
        \begin{itemize}
            \item Diferentes tamanhos de mistura ($M$)
            \item Diferentes tamanhos de características ($\boldsymbol{\Delta}$)
            \pause
        \end{itemize}
        \item Comparar verificações utilizando GMM e AGMM
    \end{itemize}
\end{description}
\end{frame}

\section{Sistemas de Reconhecimento de Locutor}
\label{sec:speaker-recognition-systems}

\contentscurrent

\begin{frame}
\frametitle{Sistemas de Reconhecimento de Locutor}
\end{frame}

\documentclass[a4paper,onecolumn,12pt]{article}

\usepackage{times}
\usepackage[utf8]{inputenc}
\usepackage[english]{babel}
\usepackage[a4paper,margin=2cm]{geometry}
\usepackage{authblk}
\usepackage{titlesec}
\usepackage[pdftex]{graphics}
\usepackage{mathtools}
\usepackage{enumitem}


\topmargin      0.0cm
\headheight     0.0cm
\headsep        0.0cm
\oddsidemargin  0.0cm
\evensidemargin 0.0cm
\textheight     22.86cm
\textwidth      16.51cm


\begin{document}

\graphicspath{{images/}}
\renewcommand{\figurename}{\small Figure}
\newcommand{\figureref}[1]{Fig. (\ref{fig:#1})}
\newcommand{\equationref}[1]{Eq. (\ref{eq:#1})}
\newcommand{\bigsum}{\displaystyle\sum}


\section{Feature Extraction}

Lorem ispum

\subsection{Cepstrum}
\subsection{Mel Frequency Cepstral Coefficients}


\begin{thebibliography}{9}
\end{thebibliography}

\end{document}

\section{Modelos de Mistura Gaussianas}
\label{sec:gmm}

\contentscurrent

\begin{frame}
\frametitle{Modelos de Misturas Gaussianas}
\end{frame}

\chapter{Experiments}
\label{ch:experiments}

This chapter details the experiments performed on the systems described in the previous chapters, contemplating from the front-end processes until the speaker modeling and the log-likelihood ratio test (see \equationref{score_of_X}). First, a description of the corpus is made. Then, explanations about the implementation are given. At last, the results are exhibited using the feature extraction process and the GMM techniques.

\section{Corpus}
\label{sec:corpus}

The database used in this work is \emph{The MIT Mobile Device Speaker Verification Corpus} (MIT-MDSCV), \refbib{Woo et. al.}{woo.park.hazen.2006}, a \textbf{corpus} designed to evaluate voice biometric systems of high mobility. All utterances were recorded using mobile devices of different models and manufacturers.

This corpus is composed of three sections. The first, named ``Enroll 1", contains 48 speakers (22 females and 26 males), each with 54 utterances (names of ice cream flavors) of 1.8 seconds average duration, and is used to train the ASR system. The utterances were recorded in three different locations (a quiet office, a mildly noisy hallway, and a busy street intersection) as well as two different microphones (the built-in internal microphone of the handheld device and an external earpiece headset) leading to 6 distinct training conditions. The second section, named ``Enroll 2", is similar to the first with a difference in the order of the spoken utterances, and is used to test the enrolled speakers. The third section, named ``Imposters" is similar to the first two, but with 40 non-enrolled speakers (17 females and 23 males), and is used to test the robustness of the ASR system. \tableref{corpus-division} summarizes the division of the corpus.

\begin{table}[h]
    \centering
    \begin{tabular}{|l|c|c|}
    \hline
    \multicolumn{1}{|c|}{{\bf Section}} & {\bf Training} & {\bf Test} \\ \hline
    Enroll 1                            & {\bf X}        & {\bf }     \\ \hline
    Enroll 2                            & {\bf }         & {\bf X}    \\ \hline
    Imposter                            & {\bf }         & {\bf X}    \\ \hline
    \end{tabular}
    \caption{Corpus divided in training and test sets.}
    \label{tab:corpus-division}
\end{table}

All utterances are recorded in uncompressed WAV files using a single channel. For each utterance record there is a correspondent text file containing pertinent information, such as speaker who produced it, microphone used, location where the utterance was recorded, the spoken message content and etc (see \tableref{utterance-info}).

Despite being a base for speaker verification systems, in this paper MIT-MDSCV is also used for speaker identification experiments. The difference is that only ``Enroll 1" and ``Enroll 2" are used, for training and test respectively. In an ideal identification system all utterances from ``Enroll 2" are correctly identified and, in an ideal verification system there are no false detection and false rejection.

\begin{table}[h]
    \centering
    \begin{tabular}{|l|c|}
    \hline
    {\bf Speaker}    & f00              \\ \hline
    {\bf Session}    & 1                \\ \hline
    {\bf List}       & female\_list\_3a \\ \hline
    {\bf Gender}     & female           \\ \hline
    {\bf Location}   & Office           \\ \hline
    {\bf Microphone} & Headset          \\ \hline
    {\bf Phrase}     & karen livescu    \\ \hline
    \end{tabular}
    \caption{Information from first utterance of first speaker of session ``Enroll 1".}
    \label{tab:utterance-info}
\end{table}

\section{Coding and Data Preparation}
\label{sec:coding-and-data-preparation}

\chapter{Conclusion and Future Studies}
\label{ch:conclusion}

TODO escrever a conclusão após terminar tudo (antes do abstract)


%BACKMATTER

\appendix
\chapter{Identification (SSFGMM)}
\label{apx:results-identify-ssfgmm}

\section{$r = 0.95$}

\begin{table}[h]
    \small
    \centering
    \begin{tabular}{|c|c|M{2cm}|M{2cm}|M{2cm}|M{2cm}|}
    \hline
    $\boldsymbol{\Delta}$ & \bf{M} & \bf{Office} & \bf{Hallway} & \bf{Intersection} & \bf{All} \\ \hline \hline & \bf{8} & 38.70 & 44.41 & 32.37 & 50.50 \\ \cline{2-6} & \bf{16} & 41.63 & 46.37 & 32.56 & 62.35 \\ \cline{2-6}
    \multirow{5}*\bf{\textbf 0} & \bf{32} & 47.72 & 48.53 & 37.46 & 68.06 \\ \cline{2-6} & \bf{64} & 43.75 & 50.31 & 37.27 & 72.80 \\ \cline{2-6} & \bf{128} & 38.62 & 42.75 & 31.06 & 72.15 \\ \hline \hline & \bf{8} & 33.37 & 31.67 & 26.35 & 44.87 \\ \cline{2-6} & \bf{16} & 41.13 & 42.32 & 26.62 & 54.71 \\ \cline{2-6}
    \multirow{5}*\bf{\textbf 1} & \bf{32} & 44.95 & 47.92 & 30.29 & 64.47 \\ \cline{2-6} & \bf{64} & 43.13 & 43.36 & 31.64 & 70.95 \\ \cline{2-6} & \bf{128} & 33.14 & 37.15 & 21.10 & 73.84 \\ \hline \hline & \bf{8} & 32.21 & 33.49 & 26.66 & 43.02 \\ \cline{2-6} & \bf{16} & 41.09 & 42.40 & 31.10 & 54.67 \\ \cline{2-6}
    \multirow{5}*\bf{\textbf 2} & \bf{32} & 46.33 & 44.14 & 31.75 & 66.78 \\ \cline{2-6} & \bf{64} & 40.93 & 43.60 & 33.53 & 72.72 \\ \cline{2-6} & \bf{128} & 39.16 & 37.89 & 23.26 & 73.53 \\ \hline
    \end{tabular}
    \caption{Speaker identification success rates and $r = 0.95$.}
    \label{tab:identify_speakers_0.95}
\end{table}


\begin{figure}[ht]
	\centering
	\includegraphics{chapters/results-identify-ssfgmm/r-095}
	\label{fig:r-095}
\end{figure}

\newpage
\section{$r = 0.99$}

\begin{table}[h]
    \centering
    \begin{tabular}{|c|c|M{2cm}|M{2cm}|M{2cm}|M{2cm}|}
    \hline
    $\boldsymbol{\Delta}$ & \bf{M} & \bf{Office} & \bf{Hallway} & \bf{Intersection} & \bf{All} \\
    \hline
    \hline
     & \bf{8} & 41.55 & 51.31 & 41.13 & 63.70 \\
    \cline{2-6}
     & \bf{16} & 47.42 & 56.13 & 45.10 & 71.64 \\
    \cline{2-6}
    \multirow{5}{*}\bf{\textbf 0} & \bf{32} & 48.73 & 56.98 & 43.83 & 78.32 \\
    \cline{2-6}
     & \bf{64} & 49.61 & 55.52 & 43.21 & 80.83 \\
    \cline{2-6}
     & \bf{128} & 47.15 & 50.69 & 38.93 & 81.13 \\
    \hline
    \hline
     & \bf{8} & 43.90 & 52.16 & 43.09 & 65.90 \\
    \cline{2-6}
     & \bf{16} & 49.31 & 58.68 & 47.22 & 76.85 \\
    \cline{2-6}
    \multirow{5}{*}\bf{\textbf 1} & \bf{32} & 52.16 & 60.42 & 48.73 & 83.37 \\
    \cline{2-6}
     & \bf{64} & 53.94 & 60.03 & 48.77 & 86.03 \\
    \cline{2-6}
     & \bf{128} & 49.88 & 54.63 & 45.83 & 87.15 \\
    \hline
    \hline
     & \bf{8} & 43.87 & 55.25 & 43.94 & 66.63 \\
    \cline{2-6}
     & \bf{16} & 49.65 & 60.61 & 48.11 & 77.97 \\
    \cline{2-6}
    \multirow{5}{*}\bf{\textbf 2} & \bf{32} & 53.28 & 62.77 & 52.20 & 84.14 \\
    \cline{2-6}
     & \bf{64} & 53.40 & 61.11 & 51.93 & 88.31 \\
    \cline{2-6}
     & \bf{128} & 50.23 & 54.17 & 46.03 & 88.43 \\
    \hline
    \end{tabular}
    \caption{Identification rates for enrolled speakers with $r = 0.99$.}
    \label{tab:identify_speakers_0.99}
\end{table}


\begin{figure}[ht]
	\centering
	\includegraphics{chapters/results-identify-ssfgmm/r-099}
	\label{fig:r-099}
\end{figure}

\newpage
\section{$r = 1.00$}

\begin{table}[h]
    \centering
    \begin{tabular}{|c|c|M{2cm}|M{2cm}|M{2cm}|M{2cm}|}
    \hline
    $\boldsymbol{\Delta}$ & \bf{M} & \bf{Office} & \bf{Hallway} & \bf{Intersection} & \bf{All} \\ \hline \hline & \bf{8} & 40.86 & 52.01 & 41.32 & 64.47 \\ \cline{2-6} & \bf{16} & 47.69 & 56.52 & 44.79 & 72.22 \\ \cline{2-6}
    \multirow{5}*\bf{\textbf 0} & \bf{32} & 49.50 & 57.72 & 47.61 & 77.74 \\ \cline{2-6} & \bf{64} & 50.00 & 57.95 & 45.68 & 81.25 \\ \cline{2-6} & \bf{128} & 48.65 & 53.43 & 42.63 & 81.67 \\ \hline \hline & \bf{8} & 44.25 & 53.97 & 45.60 & 66.94 \\ \cline{2-6} & \bf{16} & 50.42 & 62.00 & 50.54 & 78.24 \\ \cline{2-6}
    \multirow{5}*\bf{\textbf 1} & \bf{32} & 54.28 & 63.54 & 53.86 & 84.45 \\ \cline{2-6} & \bf{64} & 55.09 & 64.81 & 52.85 & 87.31 \\ \cline{2-6} & \bf{128} & 53.32 & 59.99 & 50.46 & 88.85 \\ \hline \hline & \bf{8} & 44.37 & 57.06 & 47.30 & 69.64 \\ \cline{2-6} & \bf{16} & 50.89 & 62.81 & 52.12 & 78.78 \\ \cline{2-6}
    \multirow{5}*\bf{\textbf 2} & \bf{32} & 54.90 & 65.01 & 56.29 & 86.00 \\ \cline{2-6} & \bf{64} & 56.06 & 64.70 & 56.56 & 89.16 \\ \cline{2-6} & \bf{128} & 52.55 & 60.73 & 49.58 & 90.66 \\ \hline
    \end{tabular}
    \caption{Speaker identification success rates and $r = 1.00$.}
    \label{tab:identify_speakers_1.00}
\end{table}


\begin{figure}[ht]
	\centering
	\includegraphics{chapters/results-identify-ssfgmm/r-100}
	\label{fig:r-100}
\end{figure}

\newpage
\section{$r = 1.01$}

\begin{table}[h]
    \centering
    \begin{tabular}{|c|c|M{2cm}|M{2cm}|M{2cm}|M{2cm}|}
    \hline
    $\boldsymbol{\Delta}$ & \bf{M} & \bf{Office} & \bf{Hallway} & \bf{Intersection} & \bf{All} \\ \hline \hline & \bf{8} & 40.16 & 52.51 & 43.02 & 61.69 \\ \cline{2-6} & \bf{16} & 46.88 & 57.10 & 47.80 & 71.84 \\ \cline{2-6}
    \multirow{5}*\bf{\textbf 0} & \bf{32} & 49.92 & 59.30 & 49.11 & 76.66 \\ \cline{2-6} & \bf{64} & 50.19 & 58.95 & 48.92 & 79.94 \\ \cline{2-6} & \bf{128} & 48.38 & 55.56 & 45.22 & 81.52 \\ \hline \hline & \bf{8} & 43.36 & 54.90 & 45.18 & 65.28 \\ \cline{2-6} & \bf{16} & 49.58 & 61.07 & 53.74 & 76.74 \\ \cline{2-6}
    \multirow{5}*\bf{\textbf 1} & \bf{32} & 55.02 & 66.44 & 56.64 & 83.60 \\ \cline{2-6} & \bf{64} & 56.02 & 66.28 & 56.25 & 88.00 \\ \cline{2-6} & \bf{128} & 55.17 & 62.23 & 54.32 & 89.51 \\ \hline \hline & \bf{8} & 45.10 & 53.74 & 47.22 & 66.44 \\ \cline{2-6} & \bf{16} & 50.81 & 64.31 & 53.59 & 78.05 \\ \cline{2-6}
    \multirow{5}*\bf{\textbf 2} & \bf{32} & 56.56 & 67.09 & 58.49 & 84.72 \\ \cline{2-6} & \bf{64} & 56.10 & 66.90 & 58.33 & 89.74 \\ \cline{2-6} & \bf{128} & 55.02 & 63.54 & 56.33 & 90.55 \\ \hline
    \end{tabular}
    \label{tab:identify_speakers_1.01}
\end{table}


\begin{figure}[ht]
	\centering
	\includegraphics{chapters/results-identify-ssfgmm/r-101}
	\label{fig:r-101}
\end{figure}

\newpage
\section{$r = 1.05$}

\begin{table}[h]
    \small
    \centering
    \begin{tabular}{|c|c|M{2cm}|M{2cm}|M{2cm}|M{2cm}|}
    \hline
    $\boldsymbol{\Delta}$ & \bf{M} & \bf{Office} & \bf{Hallway} & \bf{Intersection} & \bf{All} \\ \hline \hline & \bf{8} & 22.22 & 33.02 & 34.80 & 30.71 \\ \cline{2-6} & \bf{16} & 32.52 & 41.32 & 42.67 & 42.32 \\ \cline{2-6}
    \multirow{5}*\bf{\textbf 0} & \bf{32} & 40.78 & 51.20 & 48.92 & 52.70 \\ \cline{2-6} & \bf{64} & 46.68 & 56.56 & 53.51 & 62.19 \\ \cline{2-6} & \bf{128} & 49.15 & 59.57 & 55.13 & 69.91 \\ \hline \hline & \bf{8} & 18.56 & 23.88 & 26.97 & 22.15 \\ \cline{2-6} & \bf{16} & 28.20 & 39.00 & 39.78 & 32.87 \\ \cline{2-6}
    \multirow{5}*\bf{\textbf 1} & \bf{32} & 38.39 & 51.58 & 50.46 & 50.08 \\ \cline{2-6} & \bf{64} & 49.11 & 63.46 & 58.33 & 62.96 \\ \cline{2-6} & \bf{128} & 57.99 & 66.32 & 59.41 & 75.96 \\ \hline \hline & \bf{8} & 17.52 & 25.15 & 28.43 & 19.41 \\ \cline{2-6} & \bf{16} & 28.94 & 38.27 & 42.44 & 34.30 \\ \cline{2-6}
    \multirow{5}*\bf{\textbf 2} & \bf{32} & 40.74 & 50.31 & 49.38 & 49.88 \\ \cline{2-6} & \bf{64} & 50.42 & 61.23 & 58.87 & 63.77 \\ \cline{2-6} & \bf{128} & 57.68 & 67.52 & 62.00 & 77.55 \\ \hline
    \end{tabular}
    \caption{Speaker identification success rates and $r = 1.05$.}
    \label{tab:identify_speakers_1.05}
\end{table}


\begin{figure}[ht]
	\centering
	\includegraphics{chapters/results-identify-ssfgmm/r-105}
	\label{fig:r-105}
\end{figure}


\chapter{Results for Verification using SSGMM}
\label{apx:results-verify-ssgmm}

\chapter{Results for Verification using SSAGMM}
\label{apx:results-verify-ssagmm}

\printbibliography

\end{document}