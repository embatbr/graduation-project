\documentclass[12pt,twoside,openright]{report}

%\usepackage{mathptmx}
\usepackage[utf8]{inputenc}
\usepackage[english]{babel}
\usepackage[pdftex]{graphicx}
\usepackage[a4paper,width=150mm,top=25mm,bottom=25mm,bindingoffset=6mm]{geometry}
\usepackage{fancyhdr}
\usepackage[backend=bibtex,sorting=none,citestyle=numeric-comp]{biblatex}
\usepackage{mathtools}
\usepackage{amsmath}
\usepackage[font={small}]{caption}
\usepackage{bm}
\usepackage{titlesec}
\usepackage{emptypage}
\usepackage{times}


% PRE CONFIGURATIONS

\titleformat{\chapter}
  {\normalfont\LARGE\bfseries}{\thechapter.}{1em}{}
\graphicspath{{images/}}
\addbibresource{references.bib}
\setlength{\headheight}{14.5pt}
\pagestyle{fancy}
\fancyhead[RO,LE]{}
\fancyhead[LO]{\thechapter.\space\leftmark}
\fancyhead[RE]{\rightmark}
\renewcommand{\chaptermark}[1]{\markboth{\uppercase{#1}}{}}
\bibliography{references}
\DefineBibliographyStrings{english}{%
  bibliography = {References},
}

\renewcommand{\figurename}{\small Figure}
\newcommand{\figureref}[1]{Fig. \ref{fig:#1}}
\newcommand{\equationref}[1]{Eq. \ref{eq:#1}}
\newcommand{\chapterref}[1]{Chapter \ref{ch:#1}}
\newcommand{\sectionref}[1]{Section \ref{sec:#1}}
\newcommand{\appendixref}[1]{Section \ref{apx:#1}}
\newcommand{\refbib}[1]{\cite{#1}}

%MATH COMMANDS
\newcommand{\dvec}[1]{\boldsymbol{#1}}
\newcommand{\dhatvec}[1]{\boldsymbol{\hat{#1}}}
\newcommand{\prob}[1]{P(#1)}
\newcommand{\postprob}[2]{P(#1|#2)}
\newcommand{\pdf}[1]{p(#1)}
\newcommand{\postpdf}[2]{p(#1|#2)}
\newcommand{\verifytest}[2]
{
    \left\{
        \begin{array}{ll}
            \geq #1, & \text{accept } #2,\\
            < #1, & \text{reject } #2.
        \end{array}
    \right.
}
\newcommand{\verifytestB}[2]
{
    \left\{
        \begin{array}{ll}
            \geq #1, & \text{accept } #2,\\
            < #1, & \text{reject } #2,
        \end{array}
    \right.
}


%GLOBAL DEFINITIONS

\gdef\universityname{Universidade Federal de Pernambuco}
\gdef\centername{Centro de Informática}
\gdef\papertitle{Gaussian Mixture Models for Text-Independent Speaker Recognition}
\gdef\papertype{Final Term Paper}
\gdef\authorname{Eduardo Martins Barros de Albuquerque Tenório}
\gdef\advisername{Tsang Ing Ren}
\gdef\adviserfullname{Prof. Dr. \advisername}
\gdef\defensedate{March 3, 2015}


\title{
    {\includegraphics{ufpelogo.png}}
    \\
    {\large \universityname}
    \\
    {\large \centername}
    \vfill
    \textbf{\papertitle}
    \vskip\baselineskip
    {\large \papertype}
    \vfill
}
\author{\authorname}
\date{\normalsize\vfill \defensedate}


\begin{document}
\pagenumbering{gobble}
\maketitle


%FRONTMATTER

\chapter*{Declaration}
This paper is a presentation of my research work, as partial fulfillment of the requirement for the degree in Computer Engineering. Wherever contributions of others are involved, every effort is made to indicate this clearly, with due reference to the literature, and acknowledgement of collaborative research and discussions.
\vskip\baselineskip
\noindent The work was done under the guidance of \adviserfullname, at Centro de Informática, Universidade Federal de Pernambuco, Brazil.

\vskip5\baselineskip
\begin{flushright}
\rule{0.75\textwidth}{1pt}
\vskip0.5\baselineskip
Eduardo Martins Barros de Albuquerque Tenório
\end{flushright}

\vskip2\baselineskip
\noindent In my capacity as supervisor of the candidate’s paper, I certify that the above statements are true to the best of my knowledge.

\vskip5\baselineskip
\begin{flushright}
\rule{0.75\textwidth}{1pt}
\vskip0.5\baselineskip
\adviserfullname
\end{flushright}

\vfill
\begin{center}
\defensedate
\end{center}

\chapter*{Acknowledgements}
I am thankful to my family, for the support and patience during the graduation,\\
To my adviser, Tsang Ing Ren, for the guidance,\\
To Cleice Souza, for the previous readings and suggestions,\\
To Sérgio Vieira and James Lyons, for clarify many of my questions.

\chapter*{}
\vfill
\begin{flushright}
    \textit{Live long and prosper}\\
    \vskip0.5\baselineskip
    Vulcan salute
\end{flushright}
\vfill

\chapter*{Abstract}
TODO escrever o abstract após terminar tudo (após a conclusão).\\

\tableofcontents
\pagenumbering{roman}


%MIDDLEMATTER

\section{Introdução}
\label{sec:intro}

\contentscurrent

\subsection{Reconhecimento de Locutor}

\begin{frame}
\frametitle{Reconhecimento de Locutor}
\begin{description}
    \item[Identificação] Determina a identidade de um locutor dentro de um conjunto não unitário
    \pause
    \begin{itemize}
        \item 1 para N
        \item Problema de \textbf{conjunto fechado}
        \pause
    \end{itemize}
    \item[Verificação] Determina se o locutor é quem diz ser
    \pause
    \begin{itemize}
        \item 1 para 1
        \item Problema de \textbf{conjunto aberto}
        \pause
    \end{itemize}
\end{description}

\begin{figure}
    \centering
    \includegraphics[width=0.75\textwidth]{speaker-recognition}
\end{figure}
\end{frame}

\begin{frame}
\frametitle{Dependência de texto}
\begin{description}
    \item[Com] Teste $\in$ Treinamento
    \pause
    \begin{itemize}
        \item Diversos graus de dependência
        \item Teste $\not\in$ Treinamento $\implies$ Retreinamento
        \pause
    \end{itemize}
    \item[Sem] Teste $\neq$ Treinamento
    \pause
    \begin{itemize}
        \item Características não textuais
        \item Presentes em diferentes sotaques e até \emph{gibberish}
        \pause
    \end{itemize}
    \item Este trabalho é focado em \textbf{reconhecimento de locutor independente de texto}
\end{description}
\end{frame}

\subsection{Modelos de Mistura Gaussiana}

\begin{frame}
\frametitle{Modelos de Mistura Gaussiana}
\begin{description}
    \item[GMM] \textbf{Combinação} de Gaussianas
    \item[UBM] GMM gerado por diversas \textbf{locuções de fundo}
    \item[AGMM] GMM \textbf{adaptado} a partir de um UBM
    \item[FGMM] GMM \textbf{fracionário} utilizando FCM
\end{description}
\end{frame}

\subsection{Objetivos}

\begin{frame}
\frametitle{Objetivos}
\begin{description}
    \item Implementar sistemas de reconhecimento de locutor e analizar:
    \pause
    \begin{itemize}
        \item Taxas de \textbf{sucesso} para identificação
        \pause
        \begin{itemize}
            \item Diferentes tamanhos de mistura ($M$)
            \item Diferentes tamanhos de características ($\boldsymbol{\Delta}$)
            \pause
        \end{itemize}
        \item Comparar identificações utilizando GMM e FGMM
        \pause
        \item Taxas de \textbf{falsa detecção} e \textbf{falsa rejeição} para verificação
        \pause
        \begin{itemize}
            \item Diferentes tamanhos de mistura ($M$)
            \item Diferentes tamanhos de características ($\boldsymbol{\Delta}$)
            \pause
        \end{itemize}
        \item Comparar verificações utilizando GMM e AGMM
    \end{itemize}
\end{description}
\end{frame}

\chapter{Speaker Recognition Systems}
\label{ch:speaker-recognition-system}

The process of voice recognition lies on the field of pattern classification, with the speaker's utterance (a speech signal) as input for a classifier and a decision as output. This decision may be, given a speech signal $\boldsymbol{Y}$ produced by a speaker $\mathcal{S}$ and a set $\boldsymbol{\mathcal{S}} = \{\mathcal{S}_1, ..., \mathcal{S}_S\}$ of known users,

\begin{equation}
    \text{classify } \mathcal{S} \text{ as } \mathcal{S}_i \text{ if } i = \arg\max_j P(\mathcal{S}_j|\boldsymbol{Y}).
    \label{eq:decision_speaker_identification}
\end{equation}

\noindent This is a case of speaker identification and the output is a $\mathcal{S}_i$ from $\boldsymbol{\mathcal{S}}$. Another type of decision is

\begin{equation}
    \text{if } P(\mathcal{S}_i|\boldsymbol{Y}) \verifytestB{\alpha}{\mathcal{S} \text{ as } \mathcal{S}_i}
    \label{eq:decision_speaker_verification}
\end{equation}

\noindent a speaker verification decision, with a binary output, given a $\mathcal{S}$ who produced $\boldsymbol{Y}$, a claimed identity $\mathcal{S}_i$ from $\boldsymbol{\mathcal{S}}$ and a threshold of acceptance $\alpha$. This chapter (and indirectly the whole document) is about the type of decision seen in \equationref{decision_speaker_verification}.

\section{Basic Concepts}

\subsection{Utterance}

An utterance is a piece of speech produced by a speaker. It may be a word, a statement or any vocal sound. The terms \emph{utterance} and \emph{speech signal} sometimes are used interchangeably, but from herenow speech signal will be associated to an utterance recorded, digitalized and ready to be processed. An example is shown in \figureref{speech_signal}.

\begin{figure}[ht]
    \centering
    \includegraphics[width=\textwidth]{speech_signal}
    \caption{Speech signal for utterance ``karen livescu", from the MIT dataset \cite{woo.park.hazen.2006}.}
    \label{fig:speech_signal}
\end{figure}

\subsection{Features}

The raw speech signal is unfit for usage by a recognition system. For a correct processing, the unique features from the speaker's vocal tract are extracted, reducing the number of variables the system needs to deal with (leading to a simpler implementation) and performing a better evaluation (and avoiding the curse of dimensionality). Due to the stationary properties of the speech signal when analyzed in a short period of time, it is divided in overlapping frames of small and predefined length, to avoid ``loss of significancy" in the features \cite{davis.mermelstein.1980, rabiner.schafer.2007}. This extraction is executed by the MFCC algorithm, explained in details in \chapterref{feature-extraction}.

\subsection{Dependency x Independency}

When designing a speaker recognition system, one of the most important aspects to consider is the type of dependency to text it will have. In a text-dependent system the choice of what to say is made at design time, with different degrees of freedom. The testing utterance must be a subset of the training set. A simpler version may require that the same text be spoken during the model's training and testing phases, while a more sophisticated one may allow the speaker to say just a few words from a sentence or even speak them out of order. The most common acoustic model used for this system is the HMM, with the unit modeled and the number of states depending heavily on the application \cite{hebert.2008}.

\begin{figure}[ht]
    \centering
    \includegraphics[width=\textwidth]{speaker-recognition-2}
    \caption{Speaker-recognition systems for (a) identification and (b) verification \cite{reynolds.1995a}.}
    \label{fig:speaker-recognition-2}
\end{figure}

Text-independent recognition is less problematic than the previous one for several reasons. First, the designer does not need to worry about what the speaker will say, since it is a vocal sound. The recognition is performed over the unique features of each vocal tract, shown when the person speaks. Second, for being free of time constraints, a HMM of single state (i.e. a GMM) fits well for the task \cite{hebert.2008}. Third, the ability to apply text-independent verification to unconstrained speech encourages the use of audio recorded from a wide variety of sources (e.g., speaker indexing of broadcast audio or forensic matching of law-enforcement microphone recordings) \cite{reynolds.campbell.2008}.

As stated in \sectionref{speaker-recognition}, the focus of this paper is in text-independent speaker verification, and due to that it is necessary to understand what is the likelihood ratio test and how the models are trained and tested.

\section{Basic Speaker Verification Architecture}

The architecture of a speaker verification system is pretty basic. Given a speech signal from a speaker $\mathcal{S}$ who claims to be a particular speaker $\mathcal{S}_i$ from a set of enrolled speakers $\boldsymbol{\mathcal{S}} = \{\mathcal{S}_1, ..., \mathcal{S}_S\}$, the strength of the claim resides on how similar the features $\boldsymbol{X}$, extracted from the speech $\boldsymbol{Y}$ produce by $\mathcal{S}$, are to the features from $\mathcal{S}_i$ ``memorized" by the system (see \equationref{decision_speaker_verification}). However a subset of enrolled speakers may have vocal similarities, leading to a misclassification of one enrolled speaker as another (a false positive). To reduce the error rate, the system must decide not only if a speech signal came from the claimed speaker, but also if it came from a background composed of all other enrolled speakers.

\subsection{Likelihood Ratio Test}

Given the speech signal $\boldsymbol{Y}$, and assuming it was produced by only one speaker, the detection task can be restated as a basic test between two hypoteses \cite{reynolds.1995b}:

\begin{description}\itemsep0pt
    \item $H_0$: $\boldsymbol{Y}$ is from the claimed speaker $\mathcal{S}_i$;
    \item $H_1$: $\boldsymbol{Y}$ is \underline{not} from the claimed speaker $\mathcal{S}_i$.
\end{description}

\noindent The optimum test to decide which hypotesis is valid is the \textbf{likelihood ratio test} between both posterior probabilities $P(H_0|\boldsymbol{Y})$ and $P(H_1|\boldsymbol{Y})$,

\begin{equation}
    \frac{P(H_0|\boldsymbol{Y})}{P(H_1|\boldsymbol{Y})} \verifytestB{\theta}{H_0}
    \label{eq:likelihood-ratio-test}
\end{equation}

\noindent where the decision threshold for accepting or rejecting $H_0$ is $\theta$. Applying Bayes' rule

\begin{equation}
    P(H_i|\boldsymbol{Y}) = \frac{p(\boldsymbol{Y}|H_i)P(H_i)}{p(\boldsymbol{Y})},
    \label{eq:bayes-for-hypotesis}
\end{equation}

\noindent and considering all hypoteses equally probable \textit{a priori}, \equationref{likelihood-ratio-test} can be simplified to

\begin{equation}
    \frac{p(\boldsymbol{Y}|H_0)}{p(\boldsymbol{Y}|H_1)} \verifytestB{\theta}{H_0}
    \label{eq:likelihood-ratio-test-2}
\end{equation}

\noindent where $p(\boldsymbol{Y}|H_i), i = 0, 1,$ is the probability density function for the hypothesis $H_i$ evaluated for the observed speech segment $\boldsymbol{Y}$. \figureref{likelihood-ratio-detector} shows the basic components found in speaker verification systems based on likelihood ratios. The front-end processing module extracts features $\boldsymbol{X} = \{\boldsymbol{x}_1, ..., \boldsymbol{x}_T\}$ (where $\boldsymbol{x}_t$ is the feature indexed at discrete time $t \in [1, 2, ..., T]$) from the speech signal $\boldsymbol{Y}$, and feeds it to the models for the hypotesized speaker and the background. The hypoteses $H_0$ and $H_1$ are represented mathematically by models detoned $\lambda_{hyp}$ and $\lambda_{\overline{hyp}}$, respectively. The likelihood equation from \equationref{likelihood-ratio-test-2} is be better represented as

\begin{equation}
    \frac{p(\boldsymbol{X}|\lambda_{hyp})}{p(\boldsymbol{X}|\lambda_{\overline{hyp}})} \verifytest{\theta}{\mathcal{S} \text{ as } \mathcal{S}_i}
    \label{eq:likelihood-ratio-test-3}
\end{equation}

The division seen in \equationref{likelihood-ratio-test-3} can be transformed in a subtraction by the application of the logarithm function. Since the logarithm is monotonically increasing, the behavior of the likelihood ratio is maintained, and \equationref{likelihood-ratio-test-3} is replaced by the log-likelihood ratio

\begin{equation}
    \Lambda(\boldsymbol{X}) = \log p(\boldsymbol{X}|\lambda_{hyp}) - \log p(\boldsymbol{X}|\lambda_{\overline{hyp}})
    \label{eq:log-likelihood-ratio}
\end{equation}

\noindent The more likely $\mathcal{S}$ is of $\lambda_{hyp}$ and the less likely $\mathcal{S}$ is of $\lambda_{\overline{hyp}}$ easier is to accept $\mathcal{S}$ as the claimed $\mathcal{S}_i$.

\begin{figure}[ht]
    \centering
    \includegraphics[width=0.75\textwidth]{likelihood-ratio-detector}
    \caption{Likelihood ratio-based speaker detection system \cite{bimbot.et.al.2004}.}
    \label{fig:likelihood-ratio-detector}
\end{figure}

\subsection{Training Phase}

Once the features are extracted from the speech signal, they are used to train the models $\lambda_{hyp}$ and $\lambda_{\overline{hyp}}$. A high-level demonstration of the training of $\lambda_{hyp}$ (mathematical representation of $\mathcal{S}_i$) is shown in \figureref{speaker-verification-training}.

\begin{figure}[ht]
    \centering
    \includegraphics[width=\textwidth]{speaker-verification-training}
    \caption{The statistical model of $\mathcal{S}$ is created from the features $\boldsymbol{X}$ \cite{bimbot.et.al.2004}.}
    \label{fig:speaker-verification-training}
\end{figure}

\noindent Due to $\lambda_{hyp}$ be a model of $\mathcal{S}_i$, the features used to train it (i.e., estimate $p(\boldsymbol{X}|\lambda_{hyp})$) are extracted from speech signals produced by $\mathcal{S}_i$. For $\lambda_{\overline{hyp}}$ the same process is executed, considering it a representation of another speaker (i.e., the \emph{mirror} of $\mathcal{S}_i$).

The model $\lambda_{\overline{hyp}}$ is not well-defined. It is composed of the features extracted from speech signals from all other speakers except $\mathcal{S}_i$. One popular approach to deal with this difficulty is to define the likelihood of $\lambda_{\overline{hyp}}$ as

\begin{equation}
    p(\boldsymbol{X}|\lambda_{\overline{hyp}}) = \mathcal{F}[p(\boldsymbol{X}|\lambda_1), ..., p(\boldsymbol{X}|\lambda_S)],
    \label{eq:log-likelihood-ratio}
\end{equation}

\noindent where $\mathcal{F}$ is a function of all likelihoods of set $\boldsymbol{\mathcal{S}}$ (except $\mathcal{S}_i$), such as mean or maximum. In various contexts, this set of other speakers has been called likelihood ratio sets, cohorts, and background speakers \cite{reynolds.quatieri.dunn.2000}.

Another popular approach is to create only one model for $\lambda_{\overline{hyp}}$, containing features extracted from speeches of all enrolled speakers (even the claimed $\mathcal{S}_i$). The weight of $\mathcal{S}_i$ in $\lambda_{\overline{hyp}}$ is reduced due to the presence of all other speakers from $\boldsymbol{\mathcal{S}}$. This model is named \textbf{Universal Background Model} (UBM) and is denoted by $\lambda_{bkg}$. \equationref{likelihood-ratio-test-3} is rewritten as

\begin{equation}
    \frac{p(\boldsymbol{X}|\lambda_{hyp})}{p(\boldsymbol{X}|\lambda_{bkg})} \verifytest{\theta}{\mathcal{S} \text{ as } \mathcal{S}_i}
    \label{eq:likelihood-ratio-test-4}
\end{equation}

\noindent The UBM is explained in details in \chapterref{gmm}.

\subsection{Test Phase}

As seen in \equationref{likelihood-ratio-test-4}, the decision process is based on a function \emph{Score} calculated from the likelihood ratio of $p(\boldsymbol{X}|\lambda_{hyp})$ and $p(\boldsymbol{X}|\lambda_{bkg})$. Being the vector of features $\boldsymbol{X} = \{\boldsymbol{x}_1, ..., \boldsymbol{x}_T\}$, with all $\boldsymbol{x}_t$ independent of the others, the likelihood of a model $\lambda$ given $\boldsymbol{X}$ can be written as

\begin{equation}
    p(\boldsymbol{X}|\lambda) = \prod_{t=1}^T p(\boldsymbol{x}_t|\lambda).
    \label{eq:likelihood-prod}
\end{equation}

\noindent Using the logarithm function, \equationref{likelihood-prod} becomes

\begin{equation}
    \log p(\boldsymbol{X}|\lambda) = \frac{1}{T} \sum_{t=1}^T \log p(\boldsymbol{x}_t|\lambda),
    \label{eq:log-likelihood-sum}
\end{equation}

\noindent where the term $\frac{1}{T}$ is used to normalize the log-likelihood to the duration of the speech signal. That said, the likelihood ratio given by \equationref{likelihood-ratio-test-4} becomes a subtraction

\begin{equation}
    \text{\emph{Score}}(\boldsymbol{X}) = \log p(\boldsymbol{X}|\lambda_{hyp}) - \log p(\boldsymbol{X}|\lambda_{bkg}),
    \label{eq:score_of_X}
\end{equation}

\noindent with \emph{Score}$(\boldsymbol{X})$ being compared to $\log\theta$ and maintaining the same rule from \equationref{likelihood-ratio-test-4}.

\chapter{Feature Extraction}

The feature extraction process transforms the speech signal in a sequence of vectors
representing the unique characteristics of the speaker's vocal tract. According
to \autocite{wolf.1972}, an ideal characteristic must be:

\begin{itemize}\itemsep0pt
    \item of high inter-speaker and low intra-speaker variability;
    \item robust in the presence of noise and distortion;
    \item frequent and natural in the speech;
    \item easy to measure and extract;
    \item difficult to be artificially produced;
    \item not affected by health issues and long term vocal variations.
\end{itemize}


\section{The Mel Scale}


\section{Mel Frequency Cepstral Coefficient}

TODO referenciar Davis and Mermelstein \autocite{davis.mermelstein.1980},
mostrando que seus experimentos colocam o MFCC como uma técnica de representação
de características melhor que as demais (LFCC, LPC, RC e LPCC).


\section{Energy}

\chapter{Gaussian Mixture Models}
\label{ch:gmm}

In \chapterref{speaker-recognition-system} was discussed the use of a model $\lambda_j$ for each enrolled speaker to be identified, and models $\lambda_{hyp}$ and $\lambda_{bkg}$ for a claimed speaker and for the background composed of all enrolled speakers, respectively, to perform a verification process. As the features from the speech signal (discussed in \chapterref{feature-extraction}) have unknown values until the moment of extraction, it is reasonable to model the ASR to accept random values.

For all sorts of probability distributions, the Gaussian (or normal) is the one that best describes the behavior of a random variable of unknown distribution, due to the central limit theorem. Its equation for a D-dimensional space is

\begin{equation}
    \pdf{\dvec{x}} = p(\dvec{x},\dvec{\mu},\dvec{\Sigma}) = \dgaussian{x}{\mu}{\Sigma},
    \label{eq:gaussian}
\end{equation}

\noindent where $\dvec{x}$ is a $D$-dimensional input vector, $\dvec{\mu}$ is the $D$-dimensional vector of means and $\dvec{\Sigma}$ is the $D \times D$ matrix of covariances. The vector $(\dvec{x} - \dvec{\mu})'$ is the transposed of the colum-matrix $(\dvec{x} - \dvec{\mu})$.

For the speaker recognition, a weighted sum of $p_i(\dvec{x})$ is used to model the system, trying to estimate the composition that best represents the training data. This weighted sum is named Gaussian Mixture Model (GMM), \refbib{Reynolds}{reynolds.1992}, and is given by

\begin{equation}
    \postpdf{\dvec{x}}{\lambda} = \sum_{i=1}^M w_i\pdf{\dvec{x}},
    \label{eq:gaussian_mixture}
\end{equation}

\noindent where $M$ is the number of distributions used, $\sum_{i=1}^M w_i = 1$, and $\lambda = \{w_i, \dvec{\mu}_i,\dvec{\Sigma}_i\}$, for $i = 1, ..., M$. Applying \equationref{gaussian} to \equationref{gaussian_mixture}, the likelihood for the GMM is

\begin{equation}
    \postpdf{\dvec{x}}{\lambda} = \dgaussianmixture.
    \label{eq:likelihood_gmm}
\end{equation}

The idea behind use a GMM as a model for a speaker $\mathcal{S}$ is to achieve a $\lambda$ that maximizes the likelihood when applied to features $\dvec{X}$ extracted from a speech signal produced by $\mathcal{S}$. This value is found by a Maximum Likelihood Estimation (MLE) algorithm. For a sequence of T training vectors $\dvec{X} = \{\dvec{x}_1, ..., \dvec{x}_T\}$, the GMM likelihood can be written as

\begin{equation}
    \postpdf{\dvec{x}}{\lambda} = \prod_{t=1}^T \postprob{\dvec{x}_t}{\lambda}.
    \label{eq:likelihood_gmm_mle}
\end{equation}

\noindent Unfortunately, this expression is a nonlinear function of the parameters $\lambda$ and direct maximization is not possible, \refbib{Reynolds}{reynolds.1995c}, leading to estimate $\postpdf{\dvec{x}}{\lambda}$ iteratively using the Expectation-Maximization (EM) algorithm.

\section{Expectation-Maximization}

The idea of the EM algorithm is to estimate a new model $\lambda^{(j+1)}$ from a previous model $\lambda^{(j)}$, such that $\postpdf{\dvec{x}}{\lambda^{(j+1)}} \geq \postpdf{\dvec{x}}{\lambda^{(j)}}$, approximating the GMM to the training data at each iteration, until some convergence threshold is reached. The algorithm is composed of 2 steps, an expectation of the \emph{a posteriori} probabilities for each distribution $i$, and a maximization step, when the parameters $w_i$, $\dvec{\mu}_i$ and $\dvec{\Sigma}_i$ are updated. The description ahead for the steps is for a $\lambda$ with all $\dvec{\Sigma}_i$ diagonal, i.e., change the $D \times D$ matrix $\dvec{\Sigma}_i$ for a $D$-dimensional vector $\dvec{\sigma}_i$ of variances.

\subsubsection*{E-Step}

The expectation step consists of estimate the \emph{a posteriori} probabilities for each distribution $i$ and each feature vector $\dvec{x}_t$,

\begin{equation}
    \postprob{i}{\dvec{x}_t, \lambda} = \frac{w_i p_i(\dvec{x}_t)}{\sum_{k=1}^M w_k p_k(\dvec{x}_t)}.
    \label{eq:e-step-posterior}
\end{equation}

\subsubsection*{M-Step}

In the maximization step, the parameters are updated, and the algorithm guarantees that each new $\lambda$ represents the training data better than the previous ones. From \refbib{Reynolds}{reynolds.1995c}, the updates of $w_i$, $\dvec{\mu}_i$ and $\dvec{\Sigma}_i$ are given by the following equations:

\begin{equation}
    \overline{w}_i = \frac{1}{T} \sum_{t=1}^T \postprob{i}{\dvec{x}_t, \lambda},
    \label{eq:m-step-weight}
\end{equation}

\begin{equation}
    \overline{\dvec{\mu}}_i = \frac{1}{T} \frac{\sum_{t=1}^T \postprob{i}{\dvec{x}_t, \lambda} \dvec{x}_t}{\sum_{t=1}^T \postprob{i}{\dvec{x}_t, \lambda}},
    \label{eq:m-step-means}
\end{equation}

\begin{equation}
    \overline{\dvec{\sigma}}_i = \frac{1}{T} \frac{\sum_{t=1}^T \postprob{i}{\dvec{x}_t, \lambda} \dvec{x}_t^2}{\sum_{t=1}^T \postprob{i}{\dvec{x}_t, \lambda}} - \overline{\dvec{\mu}}_i^2.
    \label{eq:m-step-means}
\end{equation}
\\

This algorithm is used to train the GMMs described in sections \sectionref{speaker-identification} and \sectionref{speaker-verification} of \chapterref{speaker-recognition-system}.

\section{Universal Background Model}

\chapter{Fractional Gaussian Mixture Model}

\chapter{Conclusion and Future Studies}
\label{ch:conclusion}

TODO escrever a conclusão após terminar tudo (antes do abstract)


%BACKMATTER

\appendix
\chapter{Codes}
\label{apx:codes}

\input{chapters/appendixB}

\printbibliography

\end{document}