\documentclass[12pt]{article}

\usepackage{times}
\usepackage[utf8]{inputenc}
\usepackage[english]{babel}
\usepackage{authblk}
\usepackage{graphics}
\usepackage{epsfig}
\usepackage{amsmath}

\topmargin      0.0cm
\headheight     0.0cm
\headsep        0.0cm
\oddsidemargin  0.0cm
\evensidemargin 0.0cm
\textheight     22.86cm
\textwidth      16.51cm

\title{\textbf{Speaker Verification Using Adapted Gaussian Mixture Models}\\
\textit{Dissertation proposal}}
\author{
    \textbf{Eduardo M. B. de A. Tenório} (\textit{student})\\
    \textbf{Tsang Ing-Ren} (\textit{advisor})\\
    \small{\texttt{\{embat,tir\}@cin.ufpe.br}}
}
\affil{
    Centro de Informática\\
    Universidade Federal de Pernambuco\\
}
\date{\today}

\begin{document}
\pagestyle{plain}
\pagenumbering{arabic}
\maketitle
\pagebreak  % here finishes the cover

\begin{abstract}

This dissertation is a reproduction of MIT Lincoln Laboratory's Gaussian Mixture Model (GMM)-based speaker verification system used successfully in several NIST Speaker Recognition Evaluations (SREs) implemented by Douglas A. Reynolds. The system is built around the likelihood ratio test for verification, using simple but effective GMMs for likelihood functions, an Universal Background Model (UBM) for alternative speaker representation, and a form of Bayesian adaptation to derive speaker models from the UBM. Additionally there is an attempt to improve the system using fractional principal component analysis (FPCA), a technique never tried before in SREs.

\end{abstract}

\pagebreak  % here finishes the abstract

\tableofcontents

\pagebreak  % here finishes the table of contents

\cleardoublepage

\section{Introduction}
\label{ch:intro}

This part provides an overall introduction of your work, including
related work of your proposal.

\subsection{Related work}
\label{ch:related}

This part talks about related work of your proposal.

\section{Proposal Topic I}
\label{ch:proposal}

The content of your proposal. Each topic occupies one section, each
with their own conclusion and future work.

\section{Proposal Topic II}
\label{ch:proposal}

The content of your proposal. Each topic occupies one section, each
with their own conclusion and future work.

\section{Research plan}
\label{ch:plan}

Provide an overview of what you have done and what need to be done.

\subsection{Plan for completion of the research}

Table \ref{tab:plan} shows my plan for completion of the research.

\begin{table}[hc]
\begin{small}
\begin{center}
\begin{tabular}{lll}
Timeline & Work & Progress\\
\hline
          & XXXXXXXXXXXXXXXXXXXXXXXXXXXXXXXXXXXXX & completed\\
Nov. xxxx & XXXXXXXXXXXXXXXXXXXXXXXXXXX & ongoing\\
Jan. xxxx & Thesis writting & \\
Feb. xxxx & Thesis defense & \\
\end{tabular}
\end{center}
\end{small}
\caption{Plan for completion of my research}
\label{tab:plan}
\end{table}

Thus, I plan to defend my thesis in XXX XXXX.

\pagebreak

\begin{footnotesize}
\bibliographystyle{plain}
\bibliography{string,itu,rfc,i-d}
\end{footnotesize}

\end{document}